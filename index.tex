% Options for packages loaded elsewhere
\PassOptionsToPackage{unicode}{hyperref}
\PassOptionsToPackage{hyphens}{url}
\PassOptionsToPackage{dvipsnames,svgnames,x11names}{xcolor}
%
\documentclass[
  letterpaper,
  DIV=11,
  numbers=noendperiod]{scrreprt}

\usepackage{amsmath,amssymb}
\usepackage{iftex}
\ifPDFTeX
  \usepackage[T1]{fontenc}
  \usepackage[utf8]{inputenc}
  \usepackage{textcomp} % provide euro and other symbols
\else % if luatex or xetex
  \usepackage{unicode-math}
  \defaultfontfeatures{Scale=MatchLowercase}
  \defaultfontfeatures[\rmfamily]{Ligatures=TeX,Scale=1}
\fi
\usepackage{lmodern}
\ifPDFTeX\else  
    % xetex/luatex font selection
\fi
% Use upquote if available, for straight quotes in verbatim environments
\IfFileExists{upquote.sty}{\usepackage{upquote}}{}
\IfFileExists{microtype.sty}{% use microtype if available
  \usepackage[]{microtype}
  \UseMicrotypeSet[protrusion]{basicmath} % disable protrusion for tt fonts
}{}
\makeatletter
\@ifundefined{KOMAClassName}{% if non-KOMA class
  \IfFileExists{parskip.sty}{%
    \usepackage{parskip}
  }{% else
    \setlength{\parindent}{0pt}
    \setlength{\parskip}{6pt plus 2pt minus 1pt}}
}{% if KOMA class
  \KOMAoptions{parskip=half}}
\makeatother
\usepackage{xcolor}
\setlength{\emergencystretch}{3em} % prevent overfull lines
\setcounter{secnumdepth}{5}
% Make \paragraph and \subparagraph free-standing
\makeatletter
\ifx\paragraph\undefined\else
  \let\oldparagraph\paragraph
  \renewcommand{\paragraph}{
    \@ifstar
      \xxxParagraphStar
      \xxxParagraphNoStar
  }
  \newcommand{\xxxParagraphStar}[1]{\oldparagraph*{#1}\mbox{}}
  \newcommand{\xxxParagraphNoStar}[1]{\oldparagraph{#1}\mbox{}}
\fi
\ifx\subparagraph\undefined\else
  \let\oldsubparagraph\subparagraph
  \renewcommand{\subparagraph}{
    \@ifstar
      \xxxSubParagraphStar
      \xxxSubParagraphNoStar
  }
  \newcommand{\xxxSubParagraphStar}[1]{\oldsubparagraph*{#1}\mbox{}}
  \newcommand{\xxxSubParagraphNoStar}[1]{\oldsubparagraph{#1}\mbox{}}
\fi
\makeatother

\usepackage{color}
\usepackage{fancyvrb}
\newcommand{\VerbBar}{|}
\newcommand{\VERB}{\Verb[commandchars=\\\{\}]}
\DefineVerbatimEnvironment{Highlighting}{Verbatim}{commandchars=\\\{\}}
% Add ',fontsize=\small' for more characters per line
\usepackage{framed}
\definecolor{shadecolor}{RGB}{241,243,245}
\newenvironment{Shaded}{\begin{snugshade}}{\end{snugshade}}
\newcommand{\AlertTok}[1]{\textcolor[rgb]{0.68,0.00,0.00}{#1}}
\newcommand{\AnnotationTok}[1]{\textcolor[rgb]{0.37,0.37,0.37}{#1}}
\newcommand{\AttributeTok}[1]{\textcolor[rgb]{0.40,0.45,0.13}{#1}}
\newcommand{\BaseNTok}[1]{\textcolor[rgb]{0.68,0.00,0.00}{#1}}
\newcommand{\BuiltInTok}[1]{\textcolor[rgb]{0.00,0.23,0.31}{#1}}
\newcommand{\CharTok}[1]{\textcolor[rgb]{0.13,0.47,0.30}{#1}}
\newcommand{\CommentTok}[1]{\textcolor[rgb]{0.37,0.37,0.37}{#1}}
\newcommand{\CommentVarTok}[1]{\textcolor[rgb]{0.37,0.37,0.37}{\textit{#1}}}
\newcommand{\ConstantTok}[1]{\textcolor[rgb]{0.56,0.35,0.01}{#1}}
\newcommand{\ControlFlowTok}[1]{\textcolor[rgb]{0.00,0.23,0.31}{\textbf{#1}}}
\newcommand{\DataTypeTok}[1]{\textcolor[rgb]{0.68,0.00,0.00}{#1}}
\newcommand{\DecValTok}[1]{\textcolor[rgb]{0.68,0.00,0.00}{#1}}
\newcommand{\DocumentationTok}[1]{\textcolor[rgb]{0.37,0.37,0.37}{\textit{#1}}}
\newcommand{\ErrorTok}[1]{\textcolor[rgb]{0.68,0.00,0.00}{#1}}
\newcommand{\ExtensionTok}[1]{\textcolor[rgb]{0.00,0.23,0.31}{#1}}
\newcommand{\FloatTok}[1]{\textcolor[rgb]{0.68,0.00,0.00}{#1}}
\newcommand{\FunctionTok}[1]{\textcolor[rgb]{0.28,0.35,0.67}{#1}}
\newcommand{\ImportTok}[1]{\textcolor[rgb]{0.00,0.46,0.62}{#1}}
\newcommand{\InformationTok}[1]{\textcolor[rgb]{0.37,0.37,0.37}{#1}}
\newcommand{\KeywordTok}[1]{\textcolor[rgb]{0.00,0.23,0.31}{\textbf{#1}}}
\newcommand{\NormalTok}[1]{\textcolor[rgb]{0.00,0.23,0.31}{#1}}
\newcommand{\OperatorTok}[1]{\textcolor[rgb]{0.37,0.37,0.37}{#1}}
\newcommand{\OtherTok}[1]{\textcolor[rgb]{0.00,0.23,0.31}{#1}}
\newcommand{\PreprocessorTok}[1]{\textcolor[rgb]{0.68,0.00,0.00}{#1}}
\newcommand{\RegionMarkerTok}[1]{\textcolor[rgb]{0.00,0.23,0.31}{#1}}
\newcommand{\SpecialCharTok}[1]{\textcolor[rgb]{0.37,0.37,0.37}{#1}}
\newcommand{\SpecialStringTok}[1]{\textcolor[rgb]{0.13,0.47,0.30}{#1}}
\newcommand{\StringTok}[1]{\textcolor[rgb]{0.13,0.47,0.30}{#1}}
\newcommand{\VariableTok}[1]{\textcolor[rgb]{0.07,0.07,0.07}{#1}}
\newcommand{\VerbatimStringTok}[1]{\textcolor[rgb]{0.13,0.47,0.30}{#1}}
\newcommand{\WarningTok}[1]{\textcolor[rgb]{0.37,0.37,0.37}{\textit{#1}}}

\providecommand{\tightlist}{%
  \setlength{\itemsep}{0pt}\setlength{\parskip}{0pt}}\usepackage{longtable,booktabs,array}
\usepackage{calc} % for calculating minipage widths
% Correct order of tables after \paragraph or \subparagraph
\usepackage{etoolbox}
\makeatletter
\patchcmd\longtable{\par}{\if@noskipsec\mbox{}\fi\par}{}{}
\makeatother
% Allow footnotes in longtable head/foot
\IfFileExists{footnotehyper.sty}{\usepackage{footnotehyper}}{\usepackage{footnote}}
\makesavenoteenv{longtable}
\usepackage{graphicx}
\makeatletter
\def\maxwidth{\ifdim\Gin@nat@width>\linewidth\linewidth\else\Gin@nat@width\fi}
\def\maxheight{\ifdim\Gin@nat@height>\textheight\textheight\else\Gin@nat@height\fi}
\makeatother
% Scale images if necessary, so that they will not overflow the page
% margins by default, and it is still possible to overwrite the defaults
% using explicit options in \includegraphics[width, height, ...]{}
\setkeys{Gin}{width=\maxwidth,height=\maxheight,keepaspectratio}
% Set default figure placement to htbp
\makeatletter
\def\fps@figure{htbp}
\makeatother
% definitions for citeproc citations
\NewDocumentCommand\citeproctext{}{}
\NewDocumentCommand\citeproc{mm}{%
  \begingroup\def\citeproctext{#2}\cite{#1}\endgroup}
\makeatletter
 % allow citations to break across lines
 \let\@cite@ofmt\@firstofone
 % avoid brackets around text for \cite:
 \def\@biblabel#1{}
 \def\@cite#1#2{{#1\if@tempswa , #2\fi}}
\makeatother
\newlength{\cslhangindent}
\setlength{\cslhangindent}{1.5em}
\newlength{\csllabelwidth}
\setlength{\csllabelwidth}{3em}
\newenvironment{CSLReferences}[2] % #1 hanging-indent, #2 entry-spacing
 {\begin{list}{}{%
  \setlength{\itemindent}{0pt}
  \setlength{\leftmargin}{0pt}
  \setlength{\parsep}{0pt}
  % turn on hanging indent if param 1 is 1
  \ifodd #1
   \setlength{\leftmargin}{\cslhangindent}
   \setlength{\itemindent}{-1\cslhangindent}
  \fi
  % set entry spacing
  \setlength{\itemsep}{#2\baselineskip}}}
 {\end{list}}
\usepackage{calc}
\newcommand{\CSLBlock}[1]{\hfill\break\parbox[t]{\linewidth}{\strut\ignorespaces#1\strut}}
\newcommand{\CSLLeftMargin}[1]{\parbox[t]{\csllabelwidth}{\strut#1\strut}}
\newcommand{\CSLRightInline}[1]{\parbox[t]{\linewidth - \csllabelwidth}{\strut#1\strut}}
\newcommand{\CSLIndent}[1]{\hspace{\cslhangindent}#1}

\KOMAoption{captions}{tableheading}
\makeatletter
\@ifpackageloaded{bookmark}{}{\usepackage{bookmark}}
\makeatother
\makeatletter
\@ifpackageloaded{caption}{}{\usepackage{caption}}
\AtBeginDocument{%
\ifdefined\contentsname
  \renewcommand*\contentsname{Table of contents}
\else
  \newcommand\contentsname{Table of contents}
\fi
\ifdefined\listfigurename
  \renewcommand*\listfigurename{List of Figures}
\else
  \newcommand\listfigurename{List of Figures}
\fi
\ifdefined\listtablename
  \renewcommand*\listtablename{List of Tables}
\else
  \newcommand\listtablename{List of Tables}
\fi
\ifdefined\figurename
  \renewcommand*\figurename{Figure}
\else
  \newcommand\figurename{Figure}
\fi
\ifdefined\tablename
  \renewcommand*\tablename{Table}
\else
  \newcommand\tablename{Table}
\fi
}
\@ifpackageloaded{float}{}{\usepackage{float}}
\floatstyle{ruled}
\@ifundefined{c@chapter}{\newfloat{codelisting}{h}{lop}}{\newfloat{codelisting}{h}{lop}[chapter]}
\floatname{codelisting}{Listing}
\newcommand*\listoflistings{\listof{codelisting}{List of Listings}}
\makeatother
\makeatletter
\makeatother
\makeatletter
\@ifpackageloaded{caption}{}{\usepackage{caption}}
\@ifpackageloaded{subcaption}{}{\usepackage{subcaption}}
\makeatother

\ifLuaTeX
  \usepackage{selnolig}  % disable illegal ligatures
\fi
\usepackage{bookmark}

\IfFileExists{xurl.sty}{\usepackage{xurl}}{} % add URL line breaks if available
\urlstyle{same} % disable monospaced font for URLs
\hypersetup{
  pdftitle={Why Statistics?},
  pdfauthor={A.D. Catterson},
  colorlinks=true,
  linkcolor={blue},
  filecolor={Maroon},
  citecolor={Blue},
  urlcolor={Blue},
  pdfcreator={LaTeX via pandoc}}


\title{Why Statistics?}
\author{A.D. Catterson}
\date{2024-12-11}

\begin{document}
\maketitle

\renewcommand*\contentsname{Table of contents}
{
\hypersetup{linkcolor=}
\setcounter{tocdepth}{2}
\tableofcontents
}

\bookmarksetup{startatroot}

\chapter*{Welcome to our class.}\label{welcome-to-our-class.}
\addcontentsline{toc}{chapter}{Welcome to our class.}

\markboth{Welcome to our class.}{Welcome to our class.}

Hi! I'm your professor. My name is Arman Daniel Catterson, and thank you
for reading these words that I wrote! Below is a short video so you can
see what I look and sound like.

\phantomsection\label{vid-prof}
\url{https://youtu.be/VXxG9kxR6c0?si=vhJWQoMTMwke1d1L}

The video ``It is me, your professor''.

This textbook will also be part of the \textbf{flipped classroom}
approach that we will take this semester.

\begin{enumerate}
\def\labelenumi{\arabic{enumi}.}
\tightlist
\item
  \textbf{Before lecture} you'll read and watch some videos to be
  introduced to the content that we will then cover more deeply in
  lecture. You'll also take a short quiz (no time limit, open-note, and
  you can take as many times as you'd like) that will encourage you to
  do the readings, and let me know what topics are still confusing to
  students.
\item
  \textbf{During lecture} we will review concepts that students still
  had questions about from the readings, and spend the rest of the time
  practicing and discussing the skills you learned. We will work on the
  homework assignment together
\item
  \textbf{After lecture} you'll complete any homework that we didn't
  finish in class, and then read for the next week's lecture.
\end{enumerate}

The flipped classroom approach requires y'all to do the readings and
watch the videos before class.

The flipped classroom approach also requires me to write text and record
videos that are engaging and helpful for the three main parts of this
class.

\begin{longtable}[]{@{}
  >{\raggedright\arraybackslash}p{(\columnwidth - 2\tabcolsep) * \real{0.5000}}
  >{\raggedright\arraybackslash}p{(\columnwidth - 2\tabcolsep) * \real{0.5000}}@{}}
\toprule\noalign{}
\endhead
\bottomrule\noalign{}
\endlastfoot
Prof.~Cat & \textbf{Statistics and R :} This semester, we'll be learning
how (and why) psychologists use statistics. We'll also learn how to use
the programming language R. \\
Mickey Mouse & \textbf{Research Methods :} Everyone's favorite public
domain character is here to help navigate you through the various tools
that psychologists use, so you can use them to conduct an independent
research project! Hooray! \\
Ratty Rat & \textbf{Critical Thinking :} \\
\end{longtable}

\bookmarksetup{startatroot}

\chapter{Chapter 1 : Introduction to Statistics, Research Methods, and
R}\label{chapter-1-introduction-to-statistics-research-methods-and-r}

In this week's reading, you'll learn why psychology students like you
are required to take this course - a statistics class, a research
methods class, and an R programming class.

\section{To-Do List :}\label{to-do-list}

\begin{enumerate}
\def\labelenumi{\arabic{enumi}.}
\tightlist
\item
  Read this document and watch the four (4) videos.
\item
  Take the on-boarding survey and save the code you get at the end for
  the quiz.
\item
  Submit the code to the quiz on bCourses.
\end{enumerate}

\section{Why Statistics in
Psychology?}\label{why-statistics-in-psychology}

As y'all know, this class is a requirement for students who want to be
psychology majors. This is exciting for me (your professor), and
probably some students too. However, over the years I have learned this
can be frustrating and stressful for students who wonder
\emph{why-the-flip} they are required to take a math class when all they
just want to learn about people (or other non-human animals), you
know!??!

I agree that people (or non-human animals) are interesting. And we all
have this interest in people (or non-human animals) because we are
complex\footnote{~One of the reasons I love teaching in the Bay Area is
  y'all are hella complex.}. While people are similar in many ways, we
also differ in radical ways; from superficial features like age and
race, to more complex ways like our personality or emotions, to highly
specific behaviors such as whether all students in the class are reading
these words or not, or how bored or excited (or any emotional
experience) students are while reading these words\ldots y'all get the
idea.

Individuals differ in terms of every facet of life, and statistics and
research methods are the tools that psychologists use to try and study
those differences.

\subsection{Statistics as a Language}\label{statistics-as-a-language}

\begin{longtable}[]{@{}
  >{\raggedright\arraybackslash}p{(\columnwidth - 2\tabcolsep) * \real{0.2020}}
  >{\raggedright\arraybackslash}p{(\columnwidth - 2\tabcolsep) * \real{0.7980}}@{}}
\toprule\noalign{}
\endhead
\bottomrule\noalign{}
\endlastfoot
\includegraphics[width=0.52083in,height=\textheight]{images/stat_cat_talk_right.png}
& Statistics is a language that scientists use to describe this
complexity. While psychology uses this language to better understand
differences in people (or non-human animals), other scientific
disciplines focus on their own domains; physicists seek to understand
differences (and similarities) in matter and energy, chemists seek to
understand differences (and similarities) in elements and compounds,
botanists seek to understand differences in plants, and economists seek
to understand money. \\
\end{longtable}

\begin{itemize}
\item
  \textbf{Vocabulary.} Statistics has specific definitions for specific
  ideas, such as ``between-person'' and ``within-person variation'' (as
  described above). Statistics terms like ``standard deviation'' are
  also just vocabulary (in this case an equation to define a type of
  variation). Some of these vocabulary words are easier to understand
  and remember than others, and like all languages, sometimes people
  disagree on the definition, and sometimes misuse these words.
\item
  \textbf{Grammar and Syntax.} The way we organize words also matters
  when learning languages. Saying ``the professor graded the students``
  has a very different meaning than ``the students graded the
  professor'', even though these share the exact same words. Statistics
  (and research methods) also requires precision in the way we organize
  the ideas, terms, and processes. We'll learn more about this as we
  discuss the scientific method (a highly structured and organized
  approach to doing research), but also as we learn how to navigate
  doing data analysis.
\item
  \textbf{Cultural Immersion.} A good language class will also help
  students to understand the ways that the language is connected to
  people, places, and history (and usually food, though I'm not sure if
  there's cultural food norms abouts statistics or research methods). In
  this class, we'll think about the ways we can immerse ourselves in the
  culture of statistics and research methods, from the cultural
  practices that inform which methods or tools to use, to the ways that
  the culture of statistics and research might needlessly create
  barriers for certain types of people or studies.
\item
  \textbf{Practice and Past Experiences.} And yes, in order to gain
  fluency in a language, you need to practice! Attendance and regular
  engagement with this class will ensure that you are able to get the
  practice that you need. It's also good to note that people differ in
  terms of their past experiences with computers and math, and are
  bringing those experiences (for better or worse) with them into this
  class.
\end{itemize}

\subsection{Variables and Variation}\label{variables-and-variation}

\subsection{Prediction and Power}\label{prediction-and-power}

\section{Why R?}\label{why-r}

\subsection{Installing R}\label{installing-r}

\subsection{Navigating R}\label{navigating-r}

\subsection{Defining Variables in R}\label{defining-variables-in-r}

\section{Why Research Methods?}\label{why-research-methods}

\subsection{Psychology as a SCIENCE TM of People (and Non-Human
Animals)}\label{psychology-as-a-science-tm-of-people-and-non-human-animals}

\subsection{Defining Your Research
Questions}\label{defining-your-research-questions}

\bookmarksetup{startatroot}

\chapter{Summary}\label{summary}

In summary, this book has no content whatsoever.

\begin{Shaded}
\begin{Highlighting}[]
\DecValTok{1} \SpecialCharTok{+} \DecValTok{1}
\end{Highlighting}
\end{Shaded}

\begin{verbatim}
[1] 2
\end{verbatim}

\bookmarksetup{startatroot}

\chapter*{References}\label{references}
\addcontentsline{toc}{chapter}{References}

\markboth{References}{References}

\phantomsection\label{refs}
\begin{CSLReferences}{0}{1}
\end{CSLReferences}




\end{document}
